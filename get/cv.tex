\documentclass[10pt, letterpaper]{article}

% Packages:
\usepackage[
    ignoreheadfoot, % set margins without considering header and footer
    top=2 cm, % seperation between body and page edge from the top
    bottom=2 cm, % seperation between body and page edge from the bottom
    left=2 cm, % seperation between body and page edge from the left
    right=2 cm, % seperation between body and page edge from the right
    footskip=1.0 cm, % seperation between body and footer
    % showframe % for debugging 
]{geometry} % for adjusting page geometry
\usepackage[explicit]{titlesec} % for customizing section titles
\usepackage{tabularx} % for making tables with fixed width columns
\usepackage{array} % tabularx requires this
\usepackage[dvipsnames]{xcolor} % for coloring text
\definecolor{primaryColor}{RGB}{0, 79, 144} % define primary color
\usepackage{enumitem} % for customizing lists
\usepackage{fontawesome5} % for using icons
\usepackage{amsmath} % for math
\usepackage[
    pdftitle={Sonny HU's CV},
    pdfauthor={Sonny HU},
    pdfcreator={LaTeX with RenderCV},
    colorlinks=true,
    urlcolor=primaryColor
]{hyperref} % for links, metadata and bookmarks
\usepackage[pscoord]{eso-pic} % for floating text on the page
\usepackage{calc} % for calculating lengths
\usepackage{bookmark} % for bookmarks
\usepackage{lastpage} % for getting the total number of pages
\usepackage{changepage} % for one column entries (adjustwidth environment)
\usepackage{paracol} % for two and three column entries
\usepackage{ifthen} % for conditional statements
\usepackage{needspace} % for avoiding page brake right after the section title
\usepackage{iftex} % check if engine is pdflatex, xetex or luatex

% Ensure that generate pdf is machine readable/ATS parsable:
\ifPDFTeX
    \input{glyphtounicode}
    \pdfgentounicode=1
    \usepackage[T1]{fontenc}
    \usepackage[utf8]{inputenc}
    \usepackage{lmodern}
\fi

\usepackage[default, type1]{sourcesanspro} 

% Some settings:
\AtBeginEnvironment{adjustwidth}{\partopsep0pt} % remove space before adjustwidth environment
\pagestyle{empty} % no header or footer
\setcounter{secnumdepth}{0} % no section numbering
\setlength{\parindent}{0pt} % no indentation
\setlength{\topskip}{0pt} % no top skip
\setlength{\columnsep}{0.15cm} % set column seperation
\makeatletter
\let\ps@customFooterStyle\ps@plain % Copy the plain style to customFooterStyle
\patchcmd{\ps@customFooterStyle}{\thepage}{
    \color{gray}\textit{\small Sonny HU - Page \thepage{} of \pageref*{LastPage}}
}{}{} % replace number by desired string
\makeatother
\pagestyle{customFooterStyle}

\titleformat{\section}{
    % avoid page braking right after the section title
    \needspace{4\baselineskip}
    % make the font size of the section title large and color it with the primary color
    \Large\color{primaryColor}
}{
}{
}{
    % print bold title, give 0.15 cm space and draw a line of 0.8 pt thickness
    % from the end of the title to the end of the body
    \textbf{#1}\hspace{0.15cm}\titlerule[0.8pt]\hspace{-0.1cm}
}[] % section title formatting

\titlespacing{\section}{
    % left space:
    -1pt
}{
    % top space:
    0.3 cm
}{
    % bottom space:
    0.2 cm
} % section title spacing

% \renewcommand\labelitemi{$\vcenter{\hbox{\small$\bullet$}}$} % custom bullet points
\newenvironment{highlights}{
    \begin{itemize}[
        topsep=0.10 cm,
        parsep=0.10 cm,
        partopsep=0pt,
        itemsep=0pt,
        leftmargin=0.4 cm + 10pt
    ]
}{
    \end{itemize}
} % new environment for highlights

\newenvironment{highlightsforbulletentries}{
    \begin{itemize}[
        topsep=0.10 cm,
        parsep=0.10 cm,
        partopsep=0pt,
        itemsep=0pt,
        leftmargin=10pt
    ]
}{
    \end{itemize}
} % new environment for highlights for bullet entries


\newenvironment{onecolentry}{
    \begin{adjustwidth}{
        0.2 cm + 0.00001 cm
    }{
        0.2 cm + 0.00001 cm
    }
}{
    \end{adjustwidth}
} % new environment for one column entries

\newenvironment{twocolentry}[2][]{
    \onecolentry
    \def\secondColumn{#2}
    \setcolumnwidth{\fill, 4.5 cm}
    \begin{paracol}{2}
}{
    \switchcolumn \raggedleft \secondColumn
    \end{paracol}
    \endonecolentry
} % new environment for two column entries

\newenvironment{threecolentry}[3][]{
    \onecolentry
    \def\thirdColumn{#3}
    \setcolumnwidth{1 cm, \fill, 4.5 cm}
    \begin{paracol}{3}
    {\raggedright #2} \switchcolumn
}{
    \switchcolumn \raggedleft \thirdColumn
    \end{paracol}
    \endonecolentry
} % new environment for three column entries

\newenvironment{header}{
    \setlength{\topsep}{0pt}\par\kern\topsep\centering\color{primaryColor}\linespread{1.5}
}{
    \par\kern\topsep
} % new environment for the header

% \newcommand{\placelastupdatedtext}{% \placetextbox{<horizontal pos>}{<vertical pos>}{<stuff>}
%   \AddToShipoutPictureFG*{% Add <stuff> to current page foreground
%     \put(
%         \LenToUnit{\paperwidth-2 cm-0.2 cm+0.05cm},
%         \LenToUnit{\paperheight-1.0 cm}
%     ){\vtop{{\null}\makebox[0pt][c]{
%         \small\color{gray}\textit{Last updated in September 2024}\hspace{\widthof{Last updated in September 2024}}
%     }}}%
%   }%
% }%

% save the original href command in a new command:
\let\hrefWithoutArrow\href

% new command for external links:
\renewcommand{\href}[2]{\hrefWithoutArrow{#1}{\ifthenelse{\equal{#2}{}}{ }{#2 }\raisebox{.15ex}{\footnotesize \faExternalLink*}}}


\begin{document}
\newcommand{\AND}{\unskip
	\cleaders\copy\ANDbox\hskip\wd\ANDbox
	\ignorespaces
}
\newsavebox\ANDbox
\sbox\ANDbox{}

% \placelastupdatedtext
\begin{header}
	\fontsize{30 pt}{30 pt}
	\textbf{Sonny HU}
	
	\vspace{0.3 cm}
	
	\normalsize
	\mbox{{\footnotesize\faMapMarker*}\hspace*{0.13cm}Toulouse}%
	\kern 0.25 cm%
	\AND%
	\kern 0.25 cm%
	\mbox{\hrefWithoutArrow{mailto:sonny.hu@toulouse.miage.fr}{{\footnotesize\faEnvelope[regular]}\hspace*{0.13cm}sonny.hu@toulouse.miage.fr}}%
	\kern 0.25 cm%
	\AND%
	\kern 0.25 cm%
	\mbox{\hrefWithoutArrow{tel:+33769236872}{{\footnotesize\faPhone*}\hspace*{0.13cm}+33769236872}}%
	\kern 0.25 cm%
	\AND%
	\kern 0.25 cm%
	\mbox{\hrefWithoutArrow{https://s0nnyhu.github.io/}{{\footnotesize\faLink}\hspace*{0.13cm}s0nnyhu.github.io}}%
    \AND%
	\kern 0.25 cm%
	\mbox{{\footnotesize\faCar*}\hspace*{0.13cm}Permis B}%
\end{header}

\vspace{0.3 cm - 0.3 cm}


\section{}


\begin{onecolentry}
	Avec une expertise technique et une passion pour l'innovation, je mets mon savoir-faire technique et ma rigueur au service de solutions digitales performantes et innovantes. Engagé dans l’excellence, je vise à accompagner les entreprises dans leur transformation numérique en apportant des réponses adaptées à leurs besoins.
\end{onecolentry}


\section{Experiences}



        
\begin{twocolentry}{
		Toulouse, FR
		
		Nov 2023 – Présent
	}
	\textbf{OneStock}, Ingénieur R\&D
	\begin{highlights}
		\item Analyse et spécification des fonctionnalités
		\item Développement de connecteurs (liaison entre OneStock et des plateformes e-commerces: Shopify \& Commercetools)
		\item Correction de bugs
		\item Rédaction de documentation technique \& utilisateur
	\end{highlights}
\end{twocolentry}


\vspace{0.2 cm}

\begin{twocolentry}{
		Toulouse, FR
		
		Jan 2023 – Sep 2023
	}
	\textbf{OneStock}, Développeur Python
	\begin{highlights}
		\item Développement de nouvelles fonctionnalités à destination de diverses clients
		\item Maintien et mise à jour des modules existants
		\item Mise en place de scripts de reprise/correction de données
		\item Analyse et correction de bugs
		\item Mise à jour de la documentation sous confluence
	\end{highlights}
\end{twocolentry}

\vspace{0.2 cm}

\begin{twocolentry}{
		Toulouse, FR
		
		Nov 2021 – Nov 2022
	}
	\textbf{Kratos Communications}, Développeur Angular/Python
	\begin{highlights}
		\item Migration et mise à jour sous Angular des interfaces existantes en JAVA
		\item Développement de nouveaux modules (de configuration des stations au sol et de plannification de missions)
		\item Automatisation du déploiement de l'application sous Gitlab sur l'environnement de test et de production
		\item Rédaction de documentation utilisateur sous Sphinx
		\item Rédaction de procédure de tests
		\item Développement de scripts de traitement de donnée sous Python
		\item Mise en place d'une API sous Flask pour l'exposition des données
	\end{highlights}
\end{twocolentry}

\begin{twocolentry}{
    Toulouse, FR
    
    Sep 2019 – Sep 2021
}
\textbf{Orange}, Alternant - Chef de projet et développement d'applications web
\begin{highlights}
    \item Conception et spécifications de nouvelles fonctionnalités
    \item Développement et évolutions de modules des applications FTTH
    \item Mise à jour des commandes d'intégrations de données (ETL)
    \item Test et documentations des outils
    \item Rédaction de procédure de tests
    \item Gestion et maintien du backlog
\end{highlights}
\end{twocolentry}

% \begin{twocolentry}{
%     Toulouse, FR
    
%     Avr 2019 – Aou 2019
% }
% \textbf{Orange}, Stage - Développeur PHP/VueJS
% \begin{highlights}
%     \item Étude de la faisabilité des demandes d’évolutions
%     \item Développement et documentation des nouveaux besoins
%     \item Mise à jour des commandes d'intégrations de données (ETL)
%     \item Aide à la mise en place d’une API
%     \item Correction des bugs identifiés
% \end{highlights}
% \end{twocolentry}

\begin{twocolentry}{
    Saint Denis, La Réunion
    
    Sep 2017 – Aou 2018
}
\textbf{Data Services Integration}, Alternance - Administrateur Système et Réseau
\begin{highlights}
    \item Gestion de serveurs (Active Directory, Microsoft Exchange, Serveur IIS, Apache)
    \item Mise en place d'un outil de monitoring (Shinken) sous Debian pour superviser l'ensemble des équipements du parc informatique
    \item Mise en place d'un script sous powershell permettant la configuration des nouveaux postes de travail
    \item Support et assistance utilisateur sous GLPI
\end{highlights}
\end{twocolentry}


\section{Formation}


\begin{threecolentry}{\textbf{M2}}{
		Sep 2018 – Sep 2021
	}
	\textbf{Universite Paul Sabatier}, Cycle MIAGE (Méthodes Informatiques Appliquées à la Gestion des Entreprises)
\end{threecolentry}


\begin{threecolentry}{\textbf{L3}}{
    Sep 2017 – Sep 2018
}
\textbf{HESIP}, Licence Responsable de Projets Informatiques
\end{threecolentry}
    
\begin{threecolentry}{\textbf{BTS}}{
		Sep 2016 – Sep 2017
	}
	\textbf{Lycée Bellepierre}, SIO (Services Informatiques aux Organisations)
\end{threecolentry}



\section{Technologies}

        
\begin{onecolentry}
	\textbf{Languages:} Javascript, Typescript, Python, PHP, Java, SQL
\end{onecolentry}

\vspace{0.2 cm}

\begin{onecolentry}
	\textbf{Framework:} Angular, React, VueJS, Symfony 4
\end{onecolentry}

\vspace{0.2 cm}

\begin{onecolentry}
	\textbf{Outils \& divers:} JIRA, Confluence, GitLab, Sphinx
\end{onecolentry}


\section{Langues}

        
\begin{onecolentry}
	\textbf{Anglais:} Très bon niveau - C1
\end{onecolentry}



\section{Centre d'intérêt}

        
\begin{onecolentry}
	\textbf{} Course - Natation - Danse - Piano - Echec
    
\end{onecolentry}

\vspace{0.2 cm}


\end{document}